\documentclass{article}
\usepackage[utf8]{inputenc}
\usepackage{graphicx}

\title{Informe del Proyecto 2048}
\author{Richard Alejandro Matos Arderí \\ Mauricio Sunde Jiménez}
\date{Universidad de La Habana \\ Licenciatura en Ciencia de la Computación \\ Asignatura: Programación Declarativa}

\begin{document}

\maketitle

\section{Funcionamiento del Juego}
El juego 2048 es un juego de rompecabezas de deslizamiento. El objetivo del juego es mover las fichas numeradas en una cuadrícula para combinarlas y crear una ficha con el número 2048. Los movimientos posibles son hacia arriba, abajo, izquierda y derecha. Cuando dos fichas con el mismo número se tocan, se combinan en una sola ficha con la suma de sus valores.

\section{Uso de la Programación Declarativa y Haskell}
Este proyecto utiliza Haskell como lenguaje de programación funcional para implementar el juego 2048. Haskell es conocido por su enfoque en la programación declarativa, donde se describe qué se quiere lograr en lugar de cómo hacerlo. Esto permite un código más conciso y fácil de razonar.

\subsection{Uso de Gloss para Animaciones e Interfaz Visual}
Gloss es una biblioteca de Haskell que facilita la creación de gráficos y animaciones. En este proyecto, Gloss se utiliza para renderizar la cuadrícula del juego y las fichas, así como para manejar la interacción del usuario. La biblioteca proporciona una abstracción de alto nivel para trabajar con gráficos, lo que simplifica el desarrollo de la interfaz visual del juego.

\section{Estructura del Proyecto}
El proyecto está organizado en varios archivos y carpetas, cada uno con un papel específico:

\subsection{game/}

\subsection{app/}
\begin{itemize}
    \item \textbf{Main.hs}: Este es el punto de entrada del programa. Aquí se inicializa el juego, se configuran los bucles de eventos y comienza la ejecución.
\end{itemize}

\subsection{src/}
Esta carpeta contiene el código fuente del juego.
\begin{itemize}
    \item \textbf{Game/}: Contiene la lógica principal del juego.
    \begin{itemize}
        \item Controlador principal que gestiona el estado del juego y las interacciones entre los diferentes componentes.
        Implementa la lógica del juego, como mover y fusionar fichas, así como verificar si el jugador ha ganado o perdido.
        También representa el tablero del juego, incluyendo su inicialización y actualizaciones.
        Define la estructura de una ficha en el juego, incluyendo su valor y posición.
        Gestiona el sistema de puntuación del juego.
        También maneja la parte gráfica del juego.
    \end{itemize}
\end{itemize}

\end{document}
